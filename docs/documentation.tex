\documentclass{book}
\usepackage{../src/Theme}

% Licensed under CC BY-SA 4.0 https://creativecommons.org/licenses/by-sa/4.0/
% Copyright (c) 2018 C. H. Lay

\title{Notexbook Documentation}
\author{C. H. Lay}

\begin{document}
\maketitle
\tableofcontents

\chapter{Usage Principles}
Any \term{ordinary page} is a standalone article.
The article formats are optimized for computer science, mathematics and biology.
The purpose is to write individual documents to refer back in the future without
a need to reread from original sources.

The base level for subject titles is \term{section}.

\chapter{Theme Stylings}
\section{Terms}
Many scientific texts seem to italicize a new term whenever it first occurs.
We also follow that convention by italicizing and coloring the term.

\begin{example}
The \term{term style} is defined in \verb|\term| command.
\end{example}

\chapter{Pseudocode Language}
\term{Pseudocode} is written in a pseudolanguage which is a mix of Kotlin, Python and C.
It also grows to support new programming paradigms whenever necessary.
The core feature is to provide compact notations and readable syntax.

\end{document}
